\documentclass[10pt,a4paper]{article}

\usepackage[a4paper,left=3cm,right=3cm,top=3cm,bottom=3cm]{geometry}

\usepackage[latin1]{inputenc}
\usepackage{amsmath}
\usepackage{amsfonts}
\usepackage{amssymb}
\usepackage{graphicx}

\usepackage{natbib}

\DeclareTextFontCommand\textsfbi{\usefont{OT1}{cmss}{m}{sl}}
\DeclareMathAlphabet\mathsfbi            {OT1}{cmss}{m}{sl}

\newcommand{\dx}{\ensuremath{\Delta x}}
\newcommand{\dy}{\ensuremath{\Delta y}}

\author{Lukas Unglehrt}
\title{Documentation}

\begin{document}
	\maketitle
	
	\section{Shallow water equations}
	\subsection{Conservation form}
	The shallow water equations in conservation form are
	\begin{subequations}
		\begin{align}
		\frac{\partial h}{\partial t} + \frac{\partial (h u)}{\partial x} + \frac{\partial (h v)}{\partial y}&=0 \\
		\frac{\partial (h u)}{\partial t} +\frac{\partial (h u^2)}{\partial x} + \frac{\partial (h u v)}{\partial y}&= -\frac{g}{2}\frac{\partial h^2}{\partial x}-g h \frac{\partial z}{\partial x} - \frac{h}{\rho}\tau_{b,x} \\
		\frac{\partial (h v)}{\partial t} +\frac{\partial (h u v)}{\partial x} + \frac{\partial (h v^2)}{\partial y}&= -\frac{g}{2}\frac{\partial h^2}{\partial y}-g h \frac{\partial z}{\partial y} - \frac{h}{\rho}\tau_{b,y}
		\end{align}
	\end{subequations}
	We abbreviate this system as
	\begin{subequations}
		\begin{equation}
		\frac{\partial \boldsymbol{U}}{\partial t} + \frac{\partial \boldsymbol{F}(\boldsymbol{U})}{\partial x} + \frac{\partial \boldsymbol{G}(\boldsymbol{U})}{\partial y} = \begin{bmatrix}
		0 \\ -g h \frac{\partial z}{\partial x} - \frac{h}{\rho}\tau_{b,x} \\ -g h \frac{\partial z}{\partial y} - \frac{h}{\rho}\tau_{b,y}
		\end{bmatrix}
		\end{equation}
		with the conservative variables
		\begin{equation}
		\boldsymbol{U} = \begin{bmatrix}
		h \\ hu \\ hv
		\end{bmatrix}
		\end{equation}
		and the fluxes
		\begin{equation}
		\boldsymbol{F}(\boldsymbol{U}) = \begin{bmatrix}
		hu \\ hu^2 + \frac{1}{2}g h^2 \\ h u v
		\end{bmatrix} = \begin{bmatrix}
		U_2 \\ \frac{U_2^2}{U_1} + \frac{1}{2}g U_1^2 \\ \frac{U_2\,U_3}{U_1}
		\end{bmatrix}
		\end{equation}
		and
		\begin{equation}
		\boldsymbol{G}(\boldsymbol{U}) = \begin{bmatrix}
		hu \\ huv \\ hv^2 + \frac{1}{2}g h^2
		\end{bmatrix} = \begin{bmatrix}
		U_3  \\ \frac{U_2\,U_3}{U_1} \\ \frac{U_3^2}{U_1} + \frac{1}{2}g U_1^2
		\end{bmatrix} \,.
		\end{equation}
		\label{eq:shallow_water_equations_conservative_form}
	\end{subequations}

	\subsection{Quasi-linear form}
	Using the chain rule of differentiation, we can rewrite the system as
	\begin{subequations}
		\begin{equation}
		\frac{\partial \boldsymbol{U}}{\partial t} + \frac{\partial \boldsymbol{F}(\boldsymbol{U})}{\partial \boldsymbol{U}}\frac{\partial \boldsymbol{U}}{\partial x} + \frac{\partial \boldsymbol{G}(\boldsymbol{U})}{\partial \boldsymbol{U}}\frac{\partial \boldsymbol{U}}{\partial y} = \begin{bmatrix}
		0 \\ -g h \frac{\partial z}{\partial x} - \frac{h}{\rho}\tau_{b,x} \\ -g h \frac{\partial z}{\partial y} - \frac{h}{\rho}\tau_{b,y}
		\end{bmatrix}\,
		\end{equation}
		where the flux Jacobians result as
		\begin{equation}
		\frac{\partial \boldsymbol{F}(\boldsymbol{U})}{\partial \boldsymbol{U}} = \begin{bmatrix}
		0 & 1 & 0 \\ -\frac{U_2^2}{U_1^2} + g U_1 & 2\frac{U_2}{U_1} & 0 \\ -\frac{U_2\,U_3}{U_1^2} & \frac{U_3}{U_1} & \frac{U_2}{U_1}
		\end{bmatrix} = \begin{bmatrix}
		0 & 1 & 0 \\ gh - u^2 & 2 u & 0 \\ -u v & v & u
		\end{bmatrix} 
		\end{equation}
		and
		\begin{equation}
		\frac{\partial\boldsymbol{G}(\boldsymbol{U})}{\partial \boldsymbol{U}} = \begin{bmatrix}
		0 & 0 & 1  \\ -\frac{U_2\,U_3}{U_1^2} & \frac{U_3}{U_1} & \frac{U_2}{U_1}\\ -\frac{U_3^2}{U_1^2} + g U_1 & 0 & 2\frac{U_3}{U_1}
		\end{bmatrix} = \begin{bmatrix}
		0 & 0 & 1  \\ -u v & v & u \\ g h-v^2 & 0 & 2 v
		\end{bmatrix} \,.
		\end{equation}
		\label{eq:shallow_water_equations_quasi-linear_form}
	\end{subequations}
	We can find out about the speed and direction of information propagation by computing the eigenvalue decomposition of the flux Jacobians. The eigenvalue decompositions read with $c=\sqrt{gh}$:
	\begin{subequations}
		\begin{equation}
		\frac{\partial \boldsymbol{F}(\boldsymbol{U})}{\partial \boldsymbol{U}} = \mathsfbi{V}^{-1}  \begin{bmatrix}
		u & 0 & 0 \\ 0 & u+c & 0 \\ 0 & 0 & u-c
		\end{bmatrix}  \mathsfbi{V} \,\text{with}\,\mathsfbi{V} = \begin{bmatrix}
		0 & 1 & 1 \\ 0 & u+c & u-c \\ 1 & v & v
		\end{bmatrix} 
		\end{equation}
		and
		\begin{equation}
		\frac{\partial \boldsymbol{G}(\boldsymbol{U})}{\partial \boldsymbol{U}} = \mathsfbi{W}^{-1}  \begin{bmatrix}
		v & 0 & 0 \\ 0 & v+c & 0 \\ 0 & 0 & v-c
		\end{bmatrix}  \mathsfbi{W} \,\text{with}\,\mathsfbi{W} = \begin{bmatrix}
		0 & 1 & 1 \\ 1 & u & u \\ 0 & v+c & v-c
		\end{bmatrix} 
		\end{equation}
	\end{subequations}

	\section{A Finite-Volume discretization}
	\subsection{The discrete conservation law}
	For discretization, we choose an equidistant Cartesian grid with a collocated arrangement of variables. As primitive fields, we choose the conservative variables $h$, $hu$ and $hv$. We integrate the shallow water equations \eqref{eq:shallow_water_equations_conservative_form} over a cell $\left[x_i-\frac{\dx}{2}; x_i+\frac{\dx}{2}\right]\times\left[y_j-\frac{\dy}{2}; y_j+\frac{\dy}{2}\right]$ and we obtain:
	\begin{equation}
	\begin{split}
	& \int_{x_i-\frac{\dx}{2}}^{x_i+\frac{\dx}{2}}\int_{y_j-\frac{\dy}{2}}^{y_j+\frac{\dy}{2}} \frac{\partial \boldsymbol{U}}{\partial t} \,\mathrm{d}y\,\mathrm{d}x +	\int_{y_j-\frac{\dy}{2}}^{y_j+\frac{\dy}{2}} \left[\boldsymbol{F}\left(\boldsymbol{U}\right)\right]_{x_i-\frac{\dx}{2}}^{x_i+\frac{\dx}{2}}\,\mathrm{d}y +\int_{x_i-\frac{\dx}{2}}^{x_i+\frac{\dx}{2}}\left[\boldsymbol{G}\left(\boldsymbol{U}\right)\right]_{y_j-\frac{\dy}{2}}^{y_j+\frac{\dy}{2}} \,\mathrm{d}x \\
	=& \int_{x_i-\frac{\dx}{2}}^{x_i+\frac{\dx}{2}}\int_{y_j-\frac{\dy}{2}}^{y_j+\frac{\dy}{2}} \begin{bmatrix}
	0 \\ -g h \frac{\partial z}{\partial x} - \frac{h}{\rho}\tau_{b,x} \\ -g h \frac{\partial z}{\partial y} - \frac{h}{\rho}\tau_{b,y}
	\end{bmatrix} \,\mathrm{d}y\,\mathrm{d}x\,.
	\end{split}
	\end{equation}
	We now define the volume averaged quantities
	\begin{equation}
	\overline{\boldsymbol{U}}_{i,j} = \frac{1}{\dx\,\dy}\int_{x_i-\frac{\dx}{2}}^{x_i+\frac{\dx}{2}}\int_{y_j-\frac{\dy}{2}}^{y_j+\frac{\dy}{2}} \boldsymbol{U}\,\mathrm{d}y\,\mathrm{d}x,
	\end{equation}
	for which we obtain a system of ordinary differential equations
	\begin{equation}
	\begin{split}
	& \frac{\mathrm{d}\overline{\boldsymbol{U}}_{i,j}}{\mathrm{d}t}\,\dx\,\dy +	\int_{y_j-\frac{\dy}{2}}^{y_j+\frac{\dy}{2}} \left[\boldsymbol{F}\left(\boldsymbol{U}\right)\right]_{x_i-\frac{\dx}{2}}^{x_i+\frac{\dx}{2}}\,\mathrm{d}y +\int_{x_i-\frac{\dx}{2}}^{x_i+\frac{\dx}{2}}\left[\boldsymbol{G}\left(\boldsymbol{U}\right)\right]_{y_j-\frac{\dy}{2}}^{y_j+\frac{\dy}{2}} \,\mathrm{d}x \\
	=& \int_{x_i-\frac{\dx}{2}}^{x_i+\frac{\dx}{2}}\int_{y_j-\frac{\dy}{2}}^{y_j+\frac{\dy}{2}} \begin{bmatrix}
	0 \\ -g h \frac{\partial z}{\partial x} - \frac{h}{\rho}\tau_{b,x} \\ -g h \frac{\partial z}{\partial y} - \frac{h}{\rho}\tau_{b,y}
	\end{bmatrix} \,\mathrm{d}y\,\mathrm{d}x\,.
	\end{split}
	\end{equation}
	Also, we define the numerical fluxes
	\begin{subequations}
		\begin{align}
		\tilde{\boldsymbol{F}}_{i+\frac{1}{2},j} &= \frac{1}{\dy} \int_{y_j-\frac{\dy}{2}}^{y_j+\frac{\dy}{2}} \left[\boldsymbol{F}\left(\boldsymbol{U}\right)\right]_{x_i+\frac{\dx}{2}}\,\mathrm{d}y \\
		\tilde{\boldsymbol{G}}_{i,j+\frac{1}{2}} &= \frac{1}{\dx} \int_{x_i-\frac{\dx}{2}}^{x_i+\frac{\dx}{2}}\left[\boldsymbol{G}\left(\boldsymbol{U}\right)\right]_{y_j+\frac{\dy}{2}} \,\mathrm{d}x \,,
		\end{align}
	\end{subequations}
	so our system results as
	\begin{equation}
	\begin{split}
	& \frac{\mathrm{d}\overline{\boldsymbol{U}}_{i,j}}{\mathrm{d}t}\,\dx\,\dy +\left(\tilde{\boldsymbol{F}}_{i+\frac{1}{2},j} -\tilde{\boldsymbol{F}}_{i-\frac{1}{2},j} \right)\dy + \left( \tilde{\boldsymbol{G}}_{i,j+\frac{1}{2}}-\tilde{\boldsymbol{G}}_{i,j-\frac{1}{2}}\right)\dx \\
	=& \int_{x_i-\frac{\dx}{2}}^{x_i+\frac{\dx}{2}}\int_{y_j-\frac{\dy}{2}}^{y_j+\frac{\dy}{2}} \begin{bmatrix}
	0 \\ -g h \frac{\partial z}{\partial x} - \frac{h}{\rho}\tau_{b,x} \\ -g h \frac{\partial z}{\partial y} - \frac{h}{\rho}\tau_{b,y}
	\end{bmatrix} \,\mathrm{d}y\,\mathrm{d}x\,.
	\end{split}
	\end{equation}
	This equation is a discrete conservation law for the quantities $h$, $hu$ and $hv$ on the cell $(i,j)$. To obtain a numerical scheme, we need to approximate the source terms and the fluxes in terms of the cell-averaged quantities $\overline{\boldsymbol{U}}_{i,j}$ and choose a time integration scheme.
	

	
\end{document}